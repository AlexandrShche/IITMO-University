\begin{multicols}{2}
\noindent что она обязана повторить вызов.
Кроме того, жюри имеет право часть очков не распределять между командами вообще.

Расписание ролей команд надо составлять так, чтобы каждая команда могла вызвать каждую другую. 
Если бой ведется по 6 (или 12) задачам, то в 6 турах как раз получаются все перестановки 3 команд ($3! = 6$):
ABC, BCA, CAB, ACB, CBA, BAC.


\begin{center}
\subsubsection*{Задачи математического боя в Челябинске на Всесоюзной математической олимпиаде*)} 
\end{center}




\begin{enumerate}

\item Дана последовательность $\{a_n\}$
и функция $f$
такая, что $f(n+1)-f(n)\geqslant n+1$.
Известно, что $a_n \leqslant a_{n+1} + a_{f(n)}$  
Докажите, что можно указать такие члены $a_{i_1}, ..., a_{i_k}  $, что $ a_{i_1} + a_{i_2} + ... + a_{i_k} > 100$.

\item В сыре, имежщем форму куба $ n  \times  n  \times  n$ вырезана сферическая дырка диаметра 1. Найти минимальное число плоских разрезов, позволяющих наверняка ее обнаружить.

\item Каждая страна на плоскости состоит из одного или двух кусков. 

Докажите, что карту можно правильно раскрасить 12 цветами.


*) Эту часть статьи подготовил к печати председатель жюри математического боя Л.Г. Лиманов.


\item В треугольнике ABC построены внутренним образом равнобедренные треугольники ABC', BCA', ACB'. Доказать, что прямые СС1, BB1, 
и AA1, перпендикулярные A'B', A'C' и B'C' соответственно, прерсекаются в одной точке.

\item Для всякого n можно указать такое m, что из m человек можно выбрать n попарно знакомых или n попарно незнакомых. 

\item даны числа $a_0, a_1, a_2, ..., a_n $ , причем $a_0 = a_n = 0, a_i > 0 $ при $i\neq 0,n $ и  $\frac{a_{s-1} + a_{s+1}}{2} \geqslant a_s \cos \frac{\pi}{k} $.
Доказать, что $n \geqslant k $

\item Дана функция $f$ на отрезке $[ab]$, причем  $f + f'' > 0, f(a) = f(b) = 0, f(x)>0$ на $(ab)$.
Доказать, что $b-a \geqslant \pi $. %\footnotetext[*)]{\label{foot-1}} 

\item Пусть a и n - натуральные числа, больше 1. Доказать, что 

\[a^n - a \neq \sum \limits_{d} \frac{a^n-n}{a^d-1} \]

где суммирование ведется по некоторым делителям d числа n.

*) f'' - это вторая производная функции g. Считается что она существует во всех точках отрезка [a, b].

 %ссылка на звездочку

%\footref{foot-1} 

\end{enumerate}
\end{multicols}

\line(1,0){370}

\begin{table}[b]

\begin{center}
В "Кванте" №6 слудующие опечатки:
\end{center}

\begin{tabular}{c|c|c|c|c}

стр & 	&	& н а п е ч а т а н о & д о л ж н о  б ы т ь	\\
\hline
75 & правая колонка & 14 строка снизу & $5a \pm  \sqrt{a^2 + 4a}$ & $5a -  \sqrt{a^2 + 4a}$ \\

77 & левая колонка & 8 строка снизу & $ 20 \sqrt{21}, 20 \sqrt{21} $ & $10 \sqrt{21}, 10 \sqrt{21}$\\

77 & правая колонка & 7 строка сверху & S & 2S


\end{tabular}


\end{table}