\clearpage
\begin{multicols*}{2}
	
	\setcounter{page}{79}
	
	\noindent Вариант 2
	
	\noindent \textbf{1.} $x=h-gh^2/(2v_0^2)=15.1~{м}$.\\
	\textbf{2.} $a=g(m_2-m_1)/(m_1+m_2)=5~{м/с}^2$.\\
	\textbf{3.} $F=A/(s cos \alpha)=50~{H}$.\\
	\textbf{4.} $l=m^2v^2/(2\mu gM^2)=2~{м}$.\\
	\textbf{5.} $V=\nu RT/p=9,3~\text{м}^3$, где $R=8,31~{Дж/(моль\cdot K)}$ -- универсальная газовая постоянная. \\
	\textbf{6.} $\tau = m(c\Lambda t + \lambda)/P=5474~{с}$.\\
	\textbf{7.} $q = S\male R/(R+r)=10~мкФ$.\\
	\textbf{8.} $F = q^2/(4\pi \varepsilon _0r^2)=9\cdot 10^3~Н$.\\
	\textbf{9.} $m_1/m_2=(r_1/r_2)^2=4$.\\
	\textbf{10.} $\alpha = 30$\textdegree. \vspace{0.6cm}\\
	
	\noindent \textbf{Колейдоскоп <<Кванта>>}
	
	\textit{<<Квант>> \textnumero 2)}\vspace{3mm}
	
	\noindent \textbf{Вопросы и задачи} 
	\begin{spacing}{0.9}
		\noindent \textbf{1.} Свет, испускаемый лазером, -- почти строго параллельные лучи. \\
		\textbf{2.} Для разных длин световых волн показатели преломления вещества различны. \\
		\textbf{3.} Ближе к перпендикуляру -- красный луч, дальше всех -- фиолетовый. \\
		\textbf{4.} Для любой линзы главное фокусное расстояние больше (по модулю) для красных лучей. \\
		\textbf{5.} Зелёное. \\
		\textbf{6.} Красный, поскольку при переходе из одной среды в другую частота света, определяющая цвет лучей, не изменяется. \\
		\textbf{7.} Нет, поскольку сама интерференция -- следствие принципа суперпозиции, согласно которому фронты волн, <<проникающих>> одна в другую, взаимно не деформируются. \\
		\textbf{8.} Да, так как прямая и обратная волны когерентны. \\
		\textbf{9.} Из-за стекания воды нижняя часть пленки утолщается, а верхняя становится тоньше. Поэтому соотвествующие интерференционные полосы смещаются.\\
		\textbf{10.} Из-за дифракции на краях Луны на поверхности Земли появляется интерференционная картина. \\
		\textbf{11, 12.} Начинают сказываться дифракционные явления.\vspace{2mm}
	\end{spacing}
		
	\noindent \textbf{Микроопыт}\\
	В щель будут видны темные дифракционные полосы: чёткая полоса в центре и ряд более слабых боковых.\vspace{1cm}
	
	\noindent \textbf{<<Квант>> для младших школьников} 
	
	\textit{<<Квант>> \textnumero 2)} \vspace{2mm}
	
	\begin{spacing}{0.9}
	\noindent \textbf{1.} На весах 300 монет. \\
	
	\noindent \textbf{2.} См. рис. \ref{ris:image1}. Сумма S чисел на каждой окружности равна 12, так как $4S=36+S$. \\
	\textbf{3.} См. рис. \ref{ris:image2}. \\
	\textbf{4.} Ошибка в графе <<Разность мячей>> у команды Швеции: при одном выигрыше и одной ничьей разность мячей не может быть <<1-1>>. Общее колчисетво забитых мячей равно 11,\\
	\end{spacing}	


	\begin{figure}[H]     
		\begin{minipage}[h]{0.49\linewidth}
			
			{\includegraphics[width=0.9\linewidth]{cat.jpg}}	
			\caption{}
			\label{ris:image1} 
			
		\end{minipage}     
		\hfill
		\begin{minipage}[h]{0.49\linewidth}     
	
			{\includegraphics[width=0.9\linewidth]{cat.jpg}}
			\caption{}
			\label{ris:image2} 
			
		\end{minipage}     
		    
	\end{figure}

	\begin{spacing}{0.9}
		\noindent а число пропущенных -- 12. Поэтому ошибка в счёте на 1 мяч, т. е. разность мячей Швеции равна <<2--1>>, либо <<1--0>>. Рассмотрение этих вариантов приводит к седующей таблице:
	\end{spacing}
	
	\begin{spacing}{0.9}
		\begin{table}[H]
			\resizebox{\linewidth}{!}{
				\begin{tabular}{cccccccc}
					\hline
					\multicolumn{1}{|c|}{} & \multicolumn{1}{c|}{\scriptsize Венгр.} & \multicolumn{1}{c|}{\scriptsize Швец.} & \multicolumn{1}{c|}{\scriptsize Исп.} & \multicolumn{1}{c|}{\scriptsize Ирл.} & \multicolumn{1}{c|}{\scriptsize Франц.} & \multicolumn{1}{c|}{\scriptsize Разн.} & \multicolumn{1}{c|}{\scriptsize Оч-} \\ 
					\multicolumn{1}{|c|}{} & \multicolumn{1}{c|}{} & \multicolumn{1}{c|}{} & \multicolumn{1}{c|}{} & \multicolumn{1}{c|}{} & \multicolumn{1}{c|}{} & \multicolumn{1}{c|}{\scriptsize мяч.} & \multicolumn{1}{c|}{\scriptsize ки.} \\ \hline
					\multicolumn{1}{|c|}{} & \multicolumn{1}{c|}{} & \multicolumn{1}{c|}{} & \multicolumn{1}{c|}{} & \multicolumn{1}{c|}{} & \multicolumn{1}{c|}{} & \multicolumn{1}{c|}{} & \multicolumn{1}{c|}{}\\  
					\multicolumn{1}{|c|}{Венгрия} & \multicolumn{1}{c|}{*} & \multicolumn{1}{c|}{--} & \multicolumn{1}{c|}{--} & \multicolumn{1}{c|}{2:1} & \multicolumn{1}{c|}{2:0} & \multicolumn{1}{c|}{4--1} & \multicolumn{1}{c|}{4}\\              
					\multicolumn{1}{|c|}{Швеция} & \multicolumn{1}{c|}{--} & \multicolumn{1}{c|}{*} & \multicolumn{1}{c|}{1:1} & \multicolumn{1}{c|}{1:0} & \multicolumn{1}{c|}{--} & \multicolumn{1}{c|}{2--1} & \multicolumn{1}{c|}{3}\\              
					\multicolumn{1}{|c|}{Испания} & \multicolumn{1}{c|}{--} & \multicolumn{1}{c|}{1:1} & \multicolumn{1}{c|}{*} & \multicolumn{1}{c|}{2:2} & \multicolumn{1}{c|}{--} & \multicolumn{1}{c|}{3--3} & \multicolumn{1}{c|}{2}\\               
					\multicolumn{1}{|c|}{Ирландия} & \multicolumn{1}{c|}{1:2} & \multicolumn{1}{c|}{0:1} & \multicolumn{1}{c|}{2:2} & \multicolumn{1}{c|}{*} & \multicolumn{1}{c|}{--} & \multicolumn{1}{c|}{3--5} & \multicolumn{1}{c|}{1} \\               
					\multicolumn{1}{|c|}{Франция} & \multicolumn{1}{c|}{0:2} & \multicolumn{1}{c|}{--} & \multicolumn{1}{c|}{--} & \multicolumn{1}{c|}{--} & \multicolumn{1}{c|}{*} & \multicolumn{1}{c|}{0--2} & \multicolumn{1}{c|}{0}\\            
					\multicolumn{1}{|c|}{} & \multicolumn{1}{c|}{} & \multicolumn{1}{c|}{} & \multicolumn{1}{c|}{} & 	\multicolumn{1}{c|}{} & \multicolumn{1}{c|}{} & \multicolumn{1}{c|}{} & \multicolumn{1}{c|}{}\\  \hline   
				\end{tabular}
			}
		\end{table}

	\noindent \textbf{5.} Разрежем четырёхугольник по средним линиям и сложим полученные четырёхугольник так, чтобы вершины большого четырёхуголь-
	\end{spacing}
	
	\vspace{2.5cm}
	
	\section*{АНКЕТА 3--89}
	
	\textbf{Дорогой читатель!}
	
	\begin{spacing}{0.9}
		\noindent Ежегодно последнем номере жернала мы помещали <<Нашу анкету>>. Но нам пришло в голову, что легче, проще высказать своё мнение, что называется, по свежим следам. Поэтому мы решили помещать анкету раз в квартал.\\		
		Мы обращаемся к Вам с просьбой. Ответьте, пожалуйста, на вопросы анкеты (на те, на которые Вы хотите и можете ответить), вырежьте анкету и пришлите в редакция; на конверте напишите <<АНКЕТА 3--89>>.\\		
		Очень надеемся на обратную связь.
		
		\vspace{5mm}
		
		\begin{enumerate}
			\item Класс, в котором Вы учитесь: \hrulefill \\
				Ваша профессия(если Вы работаете): \hrulefill \\ 
				\phantom{\hspace{-1mm}} {\hrulefill} \\
				круг Ваших интересов: физика, математика, астрономия, космонавтика, информатика(подчеркните).
			\item Какие разделы журнала для Вас наиболее интересны? \hrulefill \\
				\phantom{\hspace{-1mm}} {\hrulefill} \\
		\end{enumerate}
	
	\begin{flushright}
		\scriptsize (см. с. 80)
	\end{flushright} 

	\end{spacing}
\end{multicols*}

\texttt{Если я буду писать титры к нуарному кино, они будут выглядеть именно так.}



