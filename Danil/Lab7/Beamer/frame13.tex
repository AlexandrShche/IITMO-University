\begin{frame}[t]{\large \textcolor{blue}{И}змерение количества информации}
	
	\begin{spacing}{0.5}
		\noindent \color[rgb]{0,0.7,0.4} Количество информации $\equiv$ информационная энтропия - \color{black} это численная мера непредсказуемости информации. Количество информации в некотором объекте определяется непредсказуемостью состояния, в котором находится этот объект.
	\end{spacing}
		
	\vspace{0.6cm}
	
	\noindent Пусть i (s) -- функция для измерения количеств информации в объекте s, состоящем из n незваисимых частей $s_k$, где $k$ \color[rgb]{0,0.7,0.4} \textbf{свойство меры количества информации} \color{black} \textbf{i(s)} таковы:
	
	\begin{itemize}
		\item Неотрицательность: i(s) $\geq$ 0.
		\item Принцип предопределённости: если об объекте уже всё известно, то i(s) = 0.
		\item Аддитивность: i(s) = $\sum$ i($s_k$) по всем $k$.
		\item Монотонность: i(s) монотонна при монотонном изменении вероятностей.
	\end{itemize}

\end{frame}