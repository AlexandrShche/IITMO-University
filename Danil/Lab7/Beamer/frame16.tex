\begin{frame}[t]{\large \textcolor{blue}{П}ример применения меры Хартли на практике}
	\begin{spacing}{0.5}
		\small
		\noindent \textbf{Пример 1.} Ведущий загадывает число от 1 до 64. Какое количество вопросов типа <<да-нет>> понадобится, чтобы гарантированно угадать число?
		%%%%%%%%%%%%%%%%%%%%%%%%%%%%%%%%% - Сделай больше отступ и уменьши точку %%%%%%%%%%%%%%%%%%%%%%%%%%%%%%%%%%%
		\begin{itemize}
			\item \underline{Первый} вопрос: <<Загаданное число меньше 32?>>. Ответ: <<Да>>.
			\item \underline{Второй} вопрос: <<Загаданное число меньше 16?>>. Ответ: <<Нет>>.
			\item \hspace{0.5cm} ...
			\item \underline{Шестой} вопрос (в худшем случае) точно приведёт к верному ответу.
			\item Значит, в соотвествии с мерой Хартли в загадке ведущего содержится ровно $\log_2{64}$=6 бит непредсказуемости (т.е. информации)
		\end{itemize}
		
		\noindent \textbf{Пример 2.} Ведущий держит за спиной ферзя и собирается поставить его на произвольную клетку доски. Насколько непредсказуемо его решение?
		
		\begin{itemize}
			\item Всего на доске 8x8 клеток, а цвет ферзя может быть белым или чёрным, т.е. всего возможно 8x8x2 = 128 равновероятных состояний.
			\item Значит, количество информации по Хартли равно $\log_2{128}$=7 бит.
		\end{itemize}
	\end{spacing}
\end{frame}