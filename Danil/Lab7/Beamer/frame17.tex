\begin{frame}[t]{\large \textcolor{blue}{А}нализ свойств меры Хартли}
	\begin{spacing}{0.5}
		\small
		\noindent Экспериметатор одновременно подбрасывает монету (М) и кидает игральную кость (К). Какое количество информации содержится в эксперименте (Э)?
		
		\vspace{0.6cm}
		
		\noindent \color[rgb]{0,0.7,0.4} \textbf{Аддитивность:} \color{black} \\
		\vspace{1.5mm}
		$i$(Э) = $i$(M)+$i$(K)=>$i$(12 исходов) = $i$(2 исхода)+$i$(6 исходов): $\log_x{12}$ = $\log_x{2}$+$\log_x{6}$ %\; - пробел, если что
		\vspace{2mm}
		
		\noindent \color[rgb]{0,0.7,0.4} \textbf{Неотрицательность:} \color{black} \\
		\vspace{1.5mm}
		Функция $log_x$N неотрицаьельно при любом x $>$ 1 и N $\geq$ 1.
		\vspace{2mm}
		
		\noindent \color[rgb]{0,0.7,0.4} \textbf{Монотонность:} \color{black} \\
		\vspace{1.5mm}
		С увеличением p(M) или p(K) функция i(Э) монотонно возрастает.
		\vspace{2mm}
		
		\noindent \color[rgb]{0,0.7,0.4} \textbf{Принцип предопределённости:} \color{black} \\
		\vspace{1.5mm}
		При наличии всегда только одного исхода (монета и кость с магнитом) количество информации равно нулю: $\log_x$1 + $\log_x$1 = 0.
		
	\end{spacing}	
\end{frame}