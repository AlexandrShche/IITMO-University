\begin{frame}[t]{\large \textcolor{blue}{О}пределение термина <<информатика>>}
	
	\noindent \color[rgb]{0,0.7,0.4} \textbf{Информатика} \color{black}- дисциплина, изучающая свойства и структуру информации, закономерности её создания, преобразования, накопления, передачи и использования. \\
	
	\noindent \color[rgb]{0,0.7,0.4} \textbf{Англ:} \color{black} informatics = information technology + computer science + information theory
	\vspace{1.5em}
	
	\begin{center}
		\textbf{Важные даты}
	\end{center}

	\vspace{-2mm}

	\begin{itemize}
		\item 1956 - появление термина <<информатика>> (нем. informatik, Штуйнбух)
		\item 1968 - первое упоминание в СССР (информология, Харкевич)
		\item 197X - информатика стала отдельной наукой
		\item 4 декабря - день российской информатики
	\end{itemize}

\end{frame}