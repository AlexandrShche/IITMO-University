
\begin{frame}[t]{\large \textcolor{blue}{Т}ерминология: информация и данные}
	
	\noindent Международный стандарт ISO/IEC 2382:2015 \\	
	\noindent <<Information technology - Vocabulary>> (вольный пересказ): \\

	\begin{list}{}
		{
			\setlength{\leftmargin}{12mm}
			\setlength{\itemsep}{0pt}
			\setlength{\parsep}{0pt}
		}
		\item \color[rgb]{0,0.7,0.4} \textbf{Информация} \color{black} - знания относительно фактов, событий, вещей, идей и понятий.
		\item \color[rgb]{0,0.7,0.4} \textbf{Данные} \color{black} - форма представления информации в виде, пригодном для передачи или обработки.
	\end{list}
	
	\vspace{4mm}
	
	\begin{itemize}
		\item Что есть предмет информатики: информация или данные?
		\item Как измерить информацию? Какие измерить данные? \\ Пример: <<Байкал -- самое глубокое озеро Земли>>.
	\end{itemize}
\end{frame}